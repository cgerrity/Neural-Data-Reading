% Chris's expriment utility libraries - Library reference - Overview
% Written by Christopher Thomas.

% Dummy out chapter; this is now front-matter.
\iffalse
%
\chapter{Overview}
%
\else
%
\vspace*{0.75in}
{\Huge \bfseries Overview}
\vspace*{\baselineskip}
\label{sect-over}
%
\fi

This is a set of libraries and utilities written to support ephys experiment
analyses in Thilo's lab.

This is intended to be a private project for lab-specific code that is not
specific to individual experiments. Code that lends itself to reuse outside
our lab can be migrated to public projects. Code that's experiment-specific
should be in projects associated with those experiments.

Libraries are provided in the ``\texttt{libraries}'' directory. With that
directory on path, call the \linebreak ``\texttt{addPathsExpUtilsCjt}''
function to add sub-folders.

The following sets of library functions are provided:
\begin{itemize}
%
\item \textbf{``\texttt{euAlign}''} functions (Chapter \ref{sect-align})
perform time-alignment of event lists from different sources.
%
\item \textbf{``\texttt{euFT}''} functions (Chapter \ref{sect-ft})
provide wrappers for Field Trip functions and combine commonly-used sets
of Field Trip function calls.
%
\item \textbf{``\texttt{euPlot}''} functions (Chapter \ref{sect-plot})
plot experiment data for testing. These don't make paper-quality plots.
%
\item \textbf{``\texttt{euTools}''} functions (Chapter \ref{sect-tools})
are helper functions used by specific tools and scripts.
%
\item \textbf{``\texttt{euUSE}''} functions (Chapter \ref{sect-use})
are functions for reading and interpreting USE data (event codes, SynchBox
activity, gaze data, and frame data).
%
\item \textbf{``\texttt{euUtil}''} functions (Chapter \ref{sect-util})
are utility functions that don't fit into other categories.
%
\end{itemize}

The following sample code is provided:
\begin{itemize}
%
\item \textbf{``\texttt{ft\_demo.m}''} -- Simplest practical script for
reading our lab's datasets into Field Trip. This is described in Chapter
\ref{sect-sample-demo}.
%
\end{itemize}

%
% This is the end of the file.
